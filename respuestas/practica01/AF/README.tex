\documentclass{article}

\title{Práctica 1}
\date{2019-09-20}
\author{
  Luis Alfredo Lorenzo
  \and
  Luis Fernando Barocio
}

\begin{document}
  \maketitle

  La única complicación que se nos presentó fue que Haskell no permite 
  repetir el nombre de las variables en los parametros. Eso nos hubiera
  ayudado a simplificar el código usando solamente pattern matching.

  \begin{verbatim}
    esAxL1 :: PLI -> Bool
    esAxL1 (Oimp alpha (Oimp betha alpha)) = True 
    esAxL1 phi = False
  \end{verbatim}

  De cualquier forma, le dimos la vuelta a esa limitación de la siguiente forma:

  \begin{verbatim}
    esAxL1 :: PLI -> Bool
    esAxL1 (Oimp alpha1 (Oimp betha alpha2)) = alpha1 == alpha2
    esAxL1 phi = False
  \end{verbatim}
\end{document}
