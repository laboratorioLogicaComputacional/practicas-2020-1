\documentclass{article}
% pre\'ambulo

\usepackage{lmodern}
\usepackage[T1]{fontenc}
\usepackage[spanish,activeacute]{babel}
\usepackage{mathtools}

\title{Practica 1}
\author{Leal Villegas Oscar Tadeo \\ López Soto Ramses Antonio \\Rosas Bautista Miguel Filiberto}

\begin{document}
	
	\maketitle
	\textbf{Dificultades:}
	\begin{itemize}
		\item 1.- Tuvimos problemas con las sintaxis de Oimp dado que no escribiamos de forma correcta los parametros que debe recibir.
		\item 2.- En la función esAxL3 al tratar de pasar como parametro a Bot en Oimp estabamos mal por que en realidad se tiene que pasar una variable y ser comparada con Bot para poder regresar un Booleano.
		\item 3.- En la función checkPaso olvidamos por un momento que haskell tiene una función llamada elem y checa si es elemeto de una lista, dicha función la empleamos para verificar que una formula es parte de la lista de premisas.
	\end{itemize}
\end{document}
